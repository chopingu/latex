\section{A Simple Baseline to Estimate DL Energy Consumption}
\label{sec:simple-baseline}

\looseness=-1
To calculate the energy consumption of a complete architecture, the first step is to parse its configuration and extract all layers. Modules, such as AdaptiveAvgPooling and Dropout, with a negligible contribution to the total energy consumption (Fig.~\ref{fig:layer-wise-contribution}) are discarded. We then built a separate predictor for each layer type to estimate the energy consumption for each instance of that type of layer.

\looseness=-1
The input features of each predictor vary depending on the layer type, but they are generally based on the layer parameters listed in Tab.~\ref{tab:layer-params}. Furthermore, to help the model capture potential non-linear trends in the energy consumption, we may also consider the log transformation of each parameter as an additional feature \citep{NeuralPower_DBLP:journals/corr/abs-1710-05420}. Moreover, for certain models, we decided to include the MAC count as a feature, since it encodes the layer's parameters and is commonly used as a measure of runtime, implying that it could also serve as an accurate predictor of a model's energy consumption. As we do not expect any higher-order dependencies or strong non-linear relationships we chose to use Linear/Polynomial Regression models to predict the CPU energy consumption of each layer. Additionally, these models offer superior transparency and interpretablity thanks to their simplicity. Next, due to its small scale, a MinMaxScaler is always applied to the target variable ``cpu-energy''. Furthermore, we occasionally opted to use a StandardScaler to compensate for large differences in feature scales between layer parameters and the MAC count. All models and transforms were taken from scikit-learn \footnote{\url{https://scikit-learn.org/stable/modules/classes.html\#module-sklearn.preprocessing}}.

\looseness=-1
After all the models were trained, we used them to estimate the energy of each layer in the architecture. The total energy estimate for the full architecture is then found by summing the individual layer predictions. We made this design choice because we anticipate that the complexities of the individual layers will accumulate accordingly.